\documentclass{beamer}

\usepackage[french]{babel}
\usepackage[T1]{fontenc}
\usepackage[utf8]{inputenc}
\usepackage{pgf}
\usepackage{tikz}
\usetikzlibrary{arrows,automata}
\usetikzlibrary{positioning}

\begin{document}

\title{Implémentation de l'algorithme de Strassen}
\author{Christophe Cordero et Antonin Delpeuch}
\date{\today}

\frame{\titlepage}

\begin{frame}
\frametitle{Algorithme}
\begin{block}{Principe}
Le produit de deux matrices $2n \times 2n$ peut être réalisé avec 7
produits de matrices $n \times n$ et 18 additions ou soustractions de
matrices $n \times n$.
\end{block}

\begin{block}{Complexité}
$$T(2n) = 7 T(n) + O(n^2)$$
Donc $T(n) = O(n^{\log_2 7})$, où $\log_2 7 \simeq 2.8073$
\end{block}

\end{frame}

\begin{frame}
\frametitle{Implémentation}

\end{frame}

\begin{frame}
\frametitle{Résultats}

\end{frame}

\begin{frame}
\frametitle{Problème}
On veut multiplier $n$ matrices, $M_1 M_2 \cdots M_n$.

$$
\begin{pmatrix}
a^1_{1,1} & \cdots &\cdots & a^1_{1,n}
\end{pmatrix}
\begin{pmatrix}
a^2_{1,1} & a^2_{1,2} \\
\vdots    & \vdots    \\
\vdots    & \vdots    \\
a^2_{n,1} & a^2_{n,2} 
\end{pmatrix}
\begin{pmatrix}
a^3_{1,1} & a^3_{1,2} \\
a^3_{2,1} & a^3_{2,2}
\end{pmatrix}
$$
$$
\begin{pmatrix}
a^4_{1,1} & \cdots & a^4_{1,m} \\
a^4_{2,1} & \cdots & a^4_{2,m}
\end{pmatrix}
\begin{pmatrix}
a^5_{1,1} & \cdots & a^5_{1,q} \\
\vdots    & \ddots & \vdots    \\
a^5_{m,1} & \cdots & a^5_{m,q} \\
\end{pmatrix}
$$

\end{frame}

\begin{frame}
\frametitle{Implémentation}

\end{frame}

\begin{frame}
\frametitle{Résultats}


\end{frame}


\end{document}


