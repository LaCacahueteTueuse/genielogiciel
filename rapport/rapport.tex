\documentclass[a4paper]{article}

\usepackage[french]{babel}
\usepackage[T1]{fontenc}
\usepackage[utf8]{inputenc}
\usepackage{fullpage}
\usepackage{pgf}
\usepackage{tikz}
\usetikzlibrary{arrows,automata}
\usetikzlibrary{positioning}

\begin{document}

\title{Implémentation de l'algorithme de Strassen}
\author{Christophe Cordero et Antonin Delpeuch}
\date{\today}

\maketitle

\section{Arithmétique}

Les coefficients de nos matrices sont des entiers machines : des problèmes de débordement ont donc lieu.
On fait donc nos calculs modulo une constante \texttt{MATRIX_MOD}.
Si cette constante est trop petite et que la précision des entiers machine ne permet pas de la dépasser,
on peut changer la taille des entiers manipulés en changeant une ligne dans \texttt{matrix.h} :
\begin{verbatim}
typedef int matval;
\end{verbatim}

\section{Algorithme de Strassen}


\section{Multiplication optimale d'une chaîne de matrices}

L'algorithme se déroule en deux étapes :
\begin{itemize}
    \item On calcule d'abord le coût des sous-produits avec l'algorithme dynamique.
        Le découpage optimal est conservé dans un tableau.
    \item On calcule les produits avec l'algorithme de Strassen en utilisant le découpage calculé.
\end{itemize}

\paragraph{Note} Le calcul du coût est réalisé en supposant que faire le produit de matrices
$M \in \mathcal{M}_{n,k}$ et $N \in \mathcal{M}_{k,m}$ est $nkm$, or on utilise l'algorithme de Strassen,
donc il se peut que la solution calculée ne soit pas la solution optimale.

Il faudrait donc faire nos calculs de coût en flottants et rajouter un exposant $\log_2(7)$, ce qui a un coût important.
Nous avons donc décidé de garder le calcul de coût tel quel, car il donne une bonne approximation.

TODO : trouver un exemple avec lequel la solution trouvée est sous-optimale.

\section{Interface}

Multiplication paresseuse ?
cf boost

\end{document}


